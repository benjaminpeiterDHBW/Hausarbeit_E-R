\documentclass[Thesis.tex]{subfiles}
\begin{document}

\section{Einleitung}
Die vorliegende Arbeit hat zum Ziel die Verantwortungsfrage bei der 
Verbreitung von Fake News zu untersuchen. Dabei wird auf der einen 
Seite der Ersteller der Falschinformation in Betracht gezogen, auf 
der anderen Seite hingegen der Ersteller der Künstlichen Intelligenz, 
die als Grundlade zur Erstellung besagter Fake News liegt. Aufgrund der
rasanten Entwicklung der letzten Jahre im Bereich KI wird diese immer 
präsenter in unserem Alltag. Dies wirft viele neue Fragen auf, wie zum
Beispiel: „Nach welchen Regeln müssen KI-Systeme erstellt werden um
auch diese nach unseren ethischen Vorstellungen zu gestalten und sie 
im Sinne der Menschheit zu nutzen?“. Außerdem muss auch unser Rechtssystem
an solche neuen Technologien angepasst werden um auch zukünftig 
seinen Zweck zu erfüllen. In den zukünftigen Kapiteln werden die
einzelnen Begrifflichkeiten definiert und in Zusammenhang gesetzt, 
sowie die rechtlichen und ethischen Grundlagen aufgezeigt um 
abschließend eine fundierte Antwort auf die initiale Frage geben
zu können. 

\clearpage

\end{document}