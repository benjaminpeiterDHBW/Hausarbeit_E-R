\documentclass[Thesis.tex]{subfiles}
\begin{document}
\section{Theoretischer Hintergrund und Literaturüberblik}

\subsection{Grundbegriffe und Definitionen}
Zur Analyse der rechtlichen und ethischen Gesichtspunkte der Thematik müssen wir zuerst die Grundbegriffe definieren. 
Das Europäische Parlament definiert künstliche Intelligenz wiefolgt: "Künstliche Intelligenz ist die Fähigkeit einer Maschine,
 menschliche Fähigkeiten wie logisches Denken, Lernen, Planen und Kreativität zu imitieren.
KI ermöglicht es technischen Systemen, ihre Umwelt wahrzunehmen, mit dem Wahrgenommenen umzugehen
 und Probleme zu lösen, um ein bestimmtes Ziel zu erreichen." \footcite{Parlament2020}
 Fake News ist eine englischer Begriff und bedeutet übersetzt Falschnachrichten oder Desinformation. Dazwischen ist allerdings zu unterscheiden.
  Falschinformationen sind Informationen, die aus Versehen in Umlauf gebracht werden.
  Haben die Informationen das Ziel Menschen vorsätzlich zu täuschen oder zu beeinflussen sprechen wir von Desinformationen. \footcite{Bundesregierung2023Desinformation}

\subsection{Literaturüberblick zu Fake News und KI-Verantwortlichkeit}
Es gibt verschiedene Gründe, warum Menschen Fake News verbreiten. 
Der einfache Scherz ist das harmloseste Beispiel. Hier geht es nur darum, sich selbst
 oder andere zu amüsieren.

 Es gibt aber auch solche, die ein kommerzielles Interesse verfolgen. 
 Beim Clickbaiting beispielsweise wird versucht, mit aufmerksamkeitsstarken Sätzen
  wie "Das glaubst du nicht" oder "Das haben Sie noch nie gesehen" die Aufmerksamkeit
   auf ein Produkt oder eine Dienstleistung zu lenken, um diese zu verkaufen.

Fake News werden auch mit der Verschwörungstheoretiker Scene in verbindug gebracht, hier allerdings in einem etwas anderen Kontext. 
\section{Analyse der Verantwortlichkeit bei KI-generierten Fake News}

\subsection{Rechtliche Verantwortlichkeit}

\subsection{Ethische Verantwortlichkeit und moralische Implikationen}

\subsection{Vergleichende Analyse: Entwickler oder Verbreiter}

\section{Grundlagen Künst}
\end{document}