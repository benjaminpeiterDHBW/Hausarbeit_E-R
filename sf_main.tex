\documentclass[Thesis.tex]{subfiles}
\begin{document}
\section{Theoretischer Hintergrund und Literaturüberblik}

\subsection{Grundbegriffe und Definitionen}
Zur Analyse der rechtlichen und ethischen Gesichtspunkte der Thematik müssen wir zuerst die Grundbegriffe definieren. 
Das Europäische Parlament definiert künstliche Intelligenz wiefolgt: "Künstliche Intelligenz ist die Fähigkeit einer Maschine,
 menschliche Fähigkeiten wie logisches Denken, Lernen, Planen und Kreativität zu imitieren.
KI ermöglicht es technischen Systemen, ihre Umwelt wahrzunehmen, mit dem Wahrgenommenen umzugehen
 und Probleme zu lösen, um ein bestimmtes Ziel zu erreichen." \footcite{Parlament2020}
 Fake News ist eine englischer Begriff und bedeutet übersetzt Falschnachrichten oder Desinformation. Dazwischen ist allerdings zu unterscheiden.
  Falschinformationen sind Informationen, die aus Versehen in Umlauf gebracht werden.
  Haben die Informationen das Ziel Menschen vorsätzlich zu täuschen oder zu beeinflussen sprechen wir von Desinformationen. \footcite{Bundesregierung2023Desinformation}

\subsection{Literaturüberblick zu Fake News und KI-Verantwortlichkeit}
Es gibt verschiedene Gründe, warum Menschen Fake News verbreiten. 
Der einfache Scherz ist das harmloseste Beispiel. Hier geht es nur darum, sich selbst
 oder andere zu amüsieren.

 Es gibt aber auch solche, die ein kommerzielles Interesse verfolgen. 
 Beim Clickbaiting beispielsweise wird versucht, mit aufmerksamkeitsstarken Sätzen
  wie "Das glaubst du nicht" oder "Das haben Sie noch nie gesehen" die Aufmerksamkeit
   auf ein Produkt oder eine Dienstleistung zu lenken, um diese zu verkaufen.

Fake News werden auch mit der Verschwörungstheoretiker Scene in verbindug gebracht, hier allerdings in einem etwas anderen Kontext. 
\section{Analyse der Verantwortlichkeit bei KI-generierten Fake News}
Fake News werden als erhebliche Bedrohung für eine sachliche und ausgewogene Meinungsbildung wahrgenommen.
Solche Nachrichten können in Sekundenschnelle unzählige Nutzer erreichen.
 Ist die Nachricht erst einmal im Internet, wird es sehr schwierig, den Urheber ausfindig zu machen.
  Der Schaden ist in jedem Fall irreversibel, da das Löschen der ursprünglichen Nachricht die weitere Verbreitung
   nicht verhindert und Rückrufe oft weit weniger Beachtung finden als die initiale Meldung.
   Solche Fälle zeigen die große Unvereinbarkeit zwischen der geltenden Haftungsregelung im Presse- 
   und Medienrecht und den heutigen Standards der Informationsbeschaffung, wie z.B. Social Media oder 
   dem Internet.\footcite{DBWDUmgangmitFakenews} 
  In den folgen zwei Abschnitten wird die rechtliche- und Ethische Veranwortlichkeit untersucht um eine
  Basis für den Vergleich zwischen Entwickler und Verbeiter zu haben.

\subsection{Rechtliche Verantwortlichkeit}
Um die Strafbarkeit beurteilen zu können, muss zunächst zwischen einer Behauptung
 und einer allgemeinen Falschmeldung unterschieden werden. Die Veröffentlichung
einer falschen Nachricht ohne Bezugnahme auf eine bestimmte Person oder Gruppe
ist nicht strafbar. Ein Straftatbestand wie Beleidigung, üble Nachrede oder
Verleumdung (vgl.\S 185 ff. StGB)
liegt nur dann vor, wenn Personen verunglimpft oder verleumdet
werden. Der Veröffentlicher muss bewusst die Unwahrheit gesagt haben,
mit dem  Ziel den Betroffenen zu verächtlich oder in der öffentlichen
Meinung herabzusetzen.
Liegt tatsächlich eine Beleidigung nach (vg.\S 185 StGB) ist das Strafmaß eine Geldstrafe
oder ein Freiheitsstrafe von bis zu zwei Jahren. Wenn es sich um üble Nachrede (vgl. \S 186 StGB)
handelt und der Beschuldige die Tatsachen nicht beweisen kann, ist es möglich, dass es zu einer
Freiheitsstarfe bis zu fünf Jahren kommt. Bei einer Verleumdnung, das heißt der bewussten Verbreitung von Unwahrheiten,
die das Opfer in der öffentlichen Wahrnehmung verächtlich machen können (vlg. \S 187 StGB), beträgt das Strafmaß eine Geld-
oder Freiheitsstrafe von bis zu zwei Jahren. Stellt sich die üble Nachrede oder die Verleumdnung allerdings gegen eine Person
politischen Lebens, ist das Strafmaß durch (vgl. \S 188 StGB) eine Freiheitsstrafe zwischen drei Monaten bis zu fünf Jahren vorgesehen.
Leider ist es in der heutigen Zeit oft ein Problem, dass der Urheber der Nachricht nicht 
identifiziert werden kann. In diesem Fall kann nur Anzeige gegen Unbekannt 
erstattet werden, die bei Erfolglosigkeit der Strafverfolgungsbehörden eingestellt 
wird.
Ist die Strafverfolgung erfolgreich, kann der Urheber zur Verantwortung gezogen werden, was jedoch nicht bedeutet,
dass die Nachricht gelöscht wird. Dazu muss der Betroffene einen Rechtsanspruch auf Löschung,
 Berichtigung oder Unterlassung geltend machen.\footcite{DBWDUmgangmitFakenews} %\footcite{DBWDUmgangmitFakenews}
 Zur Vermeidung von Fehlinformationen in der Presse und zur Benennung eines
Verantwortlichen gibt es Regelungen in den Pressegesetzen der Länder.
Danach muss es für jedes periodische Druckwerk, wie zum Beispiel eine Zeitung,
einen verantwortlichen Redakteur geben, der im Impressum genannt wird. Dieser hat dafür
Sorge zu tragen, dass die Druckwerke frei von strafbaren Inhalten, wie zum Beispiel Desinformation,
sind, andernfalls kann er haftbar gemacht werden.
Da die Landespressegesetze nach ihrem Wortlaut nur für periodisch Druckwerke gilt,
 gibt es eine Unklarheit, welche Presseorgane den Gesetzen unterliegen. Allerdings
steht im Rundfunkstaatsvertrag, dass Telemedien, die teilweise Inhalte aus klassischen
Pressedrucken verwenden den jornalistischen Grundsätzen entsprechen müssen. 
Internetplattformen, auf denen Nutzer Nachrichten und Kommentare veröffentlichen können,
fallen unter das Telemediengesetz. Dazu gehören Plattformen wie Facebook oder Instagram,
aber auch alle Blogs und Internetforen. Anbieter sind nach dem TMG nicht verpflichtet,
Nutzerinhalte proaktiv zu kontrollieren, sie müssen rechtswidrige Inhalte aber unverzüglich löschen,
sobald sie davon Kenntnis erlangen. Das in Deutschland geltende Zivil- und Medienrecht ist bindend,
Unternehmen können sich dem nicht mit Verweis auf eigene Nutzungsbedingungen und Standards entziehen. \footcite{DBWDUmgangmitFakenews} %\footcite{DBWDUmgangmitFakenews}
   \subsection{Ethische Verantwortlichkeit und moralische Implikationen}

\subsection{Vergleichende Analyse: Entwickler oder Verbreiter}

\section{Grundlagen Künst}
\end{document}