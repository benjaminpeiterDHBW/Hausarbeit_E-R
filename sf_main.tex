\documentclass[Thesis.tex]{subfiles}
\begin{document}
\section{Theoretischer Hintergrund und Literaturüberblik}

\subsection{Grundbegriffe und Definitionen}
Zur Analyse der rechtlichen und ethischen Gesichtspunkte der Thematik müssen wir zuerst die Grundbegriffe definieren. 
Das Europäische Parlament definiert künstliche Intelligenz wiefolgt: "Künstliche Intelligenz ist die Fähigkeit einer Maschine,
menschliche Fähigkeiten wie logisches Denken, Lernen, Planen und Kreativität zu imitieren.
\ac{KI} ermöglicht es technischen Systemen, ihre Umwelt wahrzunehmen, mit dem Wahrgenommenen umzugehen
und Probleme zu lösen, um ein bestimmtes Ziel zu erreichen." \footcite{Parlament2020}
Fake News ist eine englischer Begriff und bedeutet übersetzt Falschnachrichten oder Desinformation. Dazwischen ist allerdings zu unterscheiden.
Falschinformationen sind Informationen, die aus Versehen in Umlauf gebracht werden.
Haben die Informationen das Ziel Menschen vorsätzlich zu täuschen oder zu beeinflussen sprechen wir von Desinformationen. \footcite{Bundesregierung2023Desinformation}

\subsection{Literaturüberblick zu Fake News und KI-Verantwortlichkeit}
Es gibt verschiedene Gründe, warum Menschen Fake News verbreiten. 
Der einfache Scherz ist das harmloseste Beispiel. Hier geht es nur darum, sich selbst
oder andere zu amüsieren.

Es gibt aber auch solche, die ein kommerzielles Interesse verfolgen. 
Beim Clickbaiting beispielsweise wird versucht, mit aufmerksamkeitsstarken Sätzen
wie "Das glaubst du nicht" oder "Das haben Sie noch nie gesehen" die Aufmerksamkeit
auf ein Produkt oder eine Dienstleistung zu lenken, um diese zu verkaufen. 
Fake News werden auch mit der Verschwörungstheoretiker Scene in verbindug gebracht, hier allerdings in einem etwas anderen Kontext.

\subsection{Verschiedene Arten von Deepfakes}
Das manipulieren von Bild und Videoinhalten ist schon seit Jahren ein bekanntes Problem. Allerdings war es besonders bei Bewegbildern 
immer mit sehr hohem Aufwand und können verbunden. Die extrem schnell vorschreitende Entwicklung von Künstlicher Intelligenz der letzten
Jahre ist hat auch hier heftige spuren hinterlassen. Was in beispielsweise der Filmindustrie eine nützliches Hilfsmittel ist,
kann in anderen Bereichen schnell zu einer großen Gefahr werden. \ac{KI} ermöglicht mittlerweile fast jedermann Videos in spitzen Qualität 
zu manipulieren. Diese Technik nennt man „Deepfakes“. Der Begriff leitet sich tiefen neuronalen Netzen (englisch: deep neural networks) ab, welche ein Hauptbestandteil des Verfahrens sind.
Im folgenden werden die verschiedenen Arten aufgeführt.
Die Fälschung von Gesichtern ist dabei die bekannteste Form von Deepfakes. Hierbei wird das Gesicht einer Person im Video getauscht oder die Bewegung und Mimik verändert.
Teilweise ist es sogar möglich Gesichter fast in Echtzeit zu manipulieren.
Eine weitere Form ist die fälschung von Stimmen. Dabei kann die charakteristik einer Stimme verändert werden um eine andere Wirkung des gesagten zu erziehlen.
Es ist außerdem möglich die Stimme einer anderen Person nachzuahmen. Hierbei wird die Stimme der Zielperson analysiert und die Stimme des Angreifers angepasst.
Oft werden Gesichts und Stimmfälschungen kombiniert um eine komplette authentische Fälschung zu erzeugen.
Die dritte Form ist die Fälschung von Texten. Hierbei werden Texte geschrieben, die nicht von Menschlich erzeugten zu unterscheiden sind.
Die Gefahr hierbei sind Chatbots und Sozical Bots. Hier wird dem Opfer vorgegeben mit einer chten Person im austausch zu stehen,
obwohl im hintergrund nur eine \ac{KI} arbeitet, die teil einer Betrugsmasche ist.\footcite{DeepfakesGefahren}

\clearpage
\section{Analyse der Verantwortlichkeit bei KI-generierten Fake News}
Fake News werden als erhebliche Bedrohung für eine sachliche und ausgewogene Meinungsbildung wahrgenommen.
Solche Nachrichten können in Sekundenschnelle unzählige Nutzer erreichen.
Ist die Nachricht erst einmal im Internet, wird es sehr schwierig, den Urheber ausfindig zu machen.
Der Schaden ist in jedem Fall irreversibel, da das Löschen der ursprünglichen Nachricht die weitere Verbreitung
  nicht verhindert und Rückrufe oft weit weniger Beachtung finden als die initiale Meldung.
  Solche Fälle zeigen die große Unvereinbarkeit zwischen der geltenden Haftungsregelung im Presse- 
  und Medienrecht und den heutigen Standards der Informationsbeschaffung, wie z.B. Social Media oder 
  dem Internet.\footcite{DBWDUmgangmitFakenews} 
In den folgen zwei Abschnitten wird die rechtliche- und Ethische Veranwortlichkeit untersucht um eine
Basis für den Vergleich zwischen Entwickler und Verbeiter zu haben.

\subsection{Rechtliche Verantwortlichkeit}
\subsubsection{Rechtliche Verantwortlichkeit im Fall Fakenews}
Um die Strafbarkeit beurteilen zu können, muss zunächst zwischen einer Behauptung
 und einer allgemeinen Falschmeldung unterschieden werden. Die Veröffentlichung
einer falschen Nachricht ohne Bezugnahme auf eine bestimmte Person oder Gruppe
ist nicht strafbar. Ein Straftatbestand wie Beleidigung, üble Nachrede oder
Verleumdung (vgl.\S 185 ff. StGB)
liegt nur dann vor, wenn Personen verunglimpft oder verleumdet
werden. Der Veröffentlicher muss bewusst die Unwahrheit gesagt haben,
mit dem  Ziel den Betroffenen zu verächtlich oder in der öffentlichen
Meinung herabzusetzen.
Liegt tatsächlich eine Beleidigung nach (vg.\S 185 StGB) ist das Strafmaß eine Geldstrafe
oder ein Freiheitsstrafe von bis zu zwei Jahren. Wenn es sich um üble Nachrede (vgl. \S 186 StGB)
handelt und der Beschuldige die Tatsachen nicht beweisen kann, ist es möglich, dass es zu einer
Freiheitsstarfe bis zu fünf Jahren kommt. Bei einer Verleumdnung, das heißt der bewussten Verbreitung von Unwahrheiten,
die das Opfer in der öffentlichen Wahrnehmung verächtlich machen können (vlg. \S 187 StGB), beträgt das Strafmaß eine Geld-
oder Freiheitsstrafe von bis zu zwei Jahren. Stellt sich die üble Nachrede oder die Verleumdnung allerdings gegen eine Person
politischen Lebens, ist das Strafmaß durch (vgl. \S 188 StGB) eine Freiheitsstrafe zwischen drei Monaten bis zu fünf Jahren vorgesehen.
Leider ist es in der heutigen Zeit oft ein Problem, dass der Urheber der Nachricht nicht 
identifiziert werden kann. In diesem Fall kann nur Anzeige gegen Unbekannt 
erstattet werden, die bei Erfolglosigkeit der Strafverfolgungsbehörden eingestellt 
wird.
Ist die Strafverfolgung erfolgreich, kann der Urheber zur Verantwortung gezogen werden, was jedoch nicht bedeutet,
dass die Nachricht gelöscht wird. Dazu muss der Betroffene einen Rechtsanspruch auf Löschung,
Berichtigung oder Unterlassung geltend machen.\footcite{DBWDUmgangmitFakenews} %\footcite{DBWDUmgangmitFakenews}
Zur Vermeidung von Fehlinformationen in der Presse und zur Benennung eines
Verantwortlichen gibt es Regelungen in den Pressegesetzen der Länder.
Danach muss es für jedes periodische Druckwerk, wie zum Beispiel eine Zeitung,
einen verantwortlichen Redakteur geben, der im Impressum genannt wird. Dieser hat dafür
Sorge zu tragen, dass die Druckwerke frei von strafbaren Inhalten, wie zum Beispiel Desinformation,
sind, andernfalls kann er haftbar gemacht werden.
Da die Landespressegesetze nach ihrem Wortlaut nur für periodisch Druckwerke gilt,
gibt es eine Unklarheit, welche Presseorgane den Gesetzen unterliegen. Allerdings
steht im Rundfunkstaatsvertrag, dass Telemedien, die teilweise Inhalte aus klassischen
Pressedrucken verwenden den jornalistischen Grundsätzen entsprechen müssen. 
Internetplattformen, auf denen Nutzer Nachrichten und Kommentare veröffentlichen können,
fallen unter das Telemediengesetz. Dazu gehören Plattformen wie Facebook oder Instagram,
aber auch alle Blogs und Internetforen. Anbieter sind nach dem TMG nicht verpflichtet,
Nutzerinhalte proaktiv zu kontrollieren, sie müssen rechtswidrige Inhalte aber unverzüglich löschen,
sobald sie davon Kenntnis erlangen. Das in Deutschland geltende Zivil- und Medienrecht ist bindend,
Unternehmen können sich dem nicht mit Verweis auf eigene Nutzungsbedingungen und Standards entziehen. \footcite{DBWDUmgangmitFakenews} %\footcite{DBWDUmgangmitFakenews}

\subsubsection{Rechtliche Verantwortlichkeit im Fall KI}
Bei der Entwicklung, dem Betrieb, der Nutzung und dem Vertrieb von \ac{KI}-Systemen ist die Haftung
derzeit nach den allgemeinen Haftungsgrundsätzen des aktuellen Rechts zu ermitteln. Spezielle gesetzliche
Regelungen gitb es für diesen Sonderfall noch nicht.
Zuerst betrachen wir die Haftung nach den Bundesgsetzbuch. 
Verantwortlich für \ac{KI}-generierte Inhalte ist nach dem Bundesdatenschutzgesetz immer der Nutzer,
also derjenige, der Texte, Bilder oder Videos, die mittels \ac{KI} generiert wurden, im eigenen Namen verwendet.
Der \ac{KI}-Hersteller haftet nur, wenn die vertraglich zugesicherten Eigenschaften nicht vorhanden sind oder wenn der Hersteller
keine ausreichenden Sicherheitsvorkehrungen getroffen hat und dies zu Schäden führt.
Die \ac{KI} selber kann man nicht in Verantwortung ziehen, da sie nicht als Rechtspersönliche\ac{KI}t gilt.
Ein weiterer Anhaltspunkt ist das Produkthaftungsgesetz. Hier stellt sich zunächst die Frage,
ob ein \ac{KI}-Modell als Produkt im Sinne des § 2 ProdHaftG angesehen werden kann.
Unter dieser Annahme werden \ac{KI}-Systeme ähnlich wie herkömmliche Software behandelt.
Da sie auf einem körperlichen Datenträger gespeichert werden können, erfüllen sie die Voraussetzung
(§ 2 ProdHaftG) und sind somit "bewegliche Sachen". Darüber hinaus können sie wie Software als
Produkt (§ 90 BGB) angesehen werden. Dennoch wird Software vom Gesetzgeber nicht ausdrücklich
in den Anwendungsbereich der Produkthaftung einbezogen. Daraus lässt sich schließen, dass \ac{KI}
derzeit nicht der Haftung nach dem ProdHaftG unterliegt. Für die Zukunft ist jedoch geplant, dies zu ändern. \footcite{KIHaftung2024}
\subsection{Ethische Leitlinien}
Ethische Verantwortung bei der Entwicklung und Anwendung von \ac{KI}-Systemen ist ein zentrales Thema,
um sicherzustellen, dass technologische Innovationen dem Menschen dienen und negative Auswirkungen 
möglichst vermieden werden. Entscheidend ist hierbei, dass letztlich der Mensch - als Entwickler und 
Operator die Verantwortung trägt und nicht die Maschine selbst. Das bedeutet, dass transparente 
Kontrollmechanismen zur frühzeitigen identifikation von Problemen und prävention von Missbrauch 
und Nebenwirkungen eigeführt werden sollten. \footcite{EthikLeitlinien2018}

\subsection{Vergleichende Analyse: Entwickler oder Verbreiter}
Die rechtliche Verantwortung für die Erstellung und Verbreitung von Fake News liegt in erster Linie
beim Verfasser bzw. Verbreiter der Falschinformationen und nicht beim Entwickler der \ac{KI}, mit der diese
generiert wurden. Der entscheidende Unterschied besteht darin, dass der Ersteller von Fake News bewusst
falsche oder irreführende Inhalte verbreitet, insbesondere wenn diese gegen bestimmte Personen 
oder Gruppen gerichtet sind. In solchen Fällen greifen strafrechtliche Normen wie Beleidigung,
üble Nachrede oder Verleumdung, jedoch gibt es noch keine Gesetze die speziell für den Fall von
Fake News gelten.
Der Entwickler der \ac{KI} haftet in der Regel nur dann, wenn er fahrlässig oder vorsätzlich Sicherheitslücken
oder fehlende Schutzmechanismen in seiner Software zulässt, die einen Missbrauch einfach ermöglichen.
Eine unmittelbare Haftung für die erstellten Inhalte mithilfe der \ac{KI} besteht nicht, da derjenige, der 
die \ac{KI} einsetzt und die Inhalte veröffentlicht, nach geltendem Recht als Verantwortlicher gilt. Dies entspricht
auch der allgemeinen Behandlung von Softwareprodukten, die rechtlich nicht als eigenständige Einheiten angesehen werden.
Wie schon erwähnt ist die aktuelle Gesetzgebung nicht mehr Zeitgemäß und bedarf einer Veränderung. Während
Fake News, die von Menschen erstellt werden, bereits umfassend im Strafrecht geregelt sind, bleibt der Umgang
mit \ac{KI}-generierten Falschinformationen bislang weitgehend unklar. Zwar gibt es Bestrebungen, \ac{KI}-spezifische
Haftungsregelungen zu schaffen, diese befinden sich aber noch in der Entwicklung. In Zukunft könnte eine 
strengere Regulierung dazu führen, dass \ac{KI}-Entwickler bei nachweisbarem Fehlverhalten stärker zur 
Verantwortung gezogen werden,. wenn sie bewusst Systeme zur Verfügung stellen, die es einfach möglich
machen Fake News zu erstellen und dabei keine Sicherheitssysteme besitzt um dies zu verhindern.
Vor dem Hintergrund der aktuellen Rechtslage ist anzumerken, dass die Hauptverantwortung für Fake News nach
wie vor beim jeweiligen Nutzer liegt, der sie veröffentlicht oder verbreitet. Dies verdeutlicht die Notwendigkeit 
einer kritischen Medienkompetenz sowie wirksamer Mechanismen zur Erkennung und Bekämpfung von Falschmeldungen.

\clearpage
\section{Fazit}
Abschließend lässt sich sagen, dass die \ac{KI} in den letzten Jahren immer mehr an
Bedeutung gewonnen hat. Sie beeinflusst bereits die meisten Bereiche unseres 
Lebens, auch wenn wir uns dessen nicht direkt bewusst sind. Damit gehen aber auch 
viele Gefahren einher, denen man sich stellen und denen man mit den richtigen
Maßnahmen begegnen muss.
Fake News und Deepfakes gehören zu diesen Bedrohungen, die ein großes Hindernis 
für die freie Meinungsbildung in der Demokratie darstellen. Während diese Technologien
immer besser und zugänglicher werden, hinken das Rechtssystem und die technischen
Gegenmaßnahmen hinterher.
Die Untersuchung der Verantwortlichkeit für die Verbreitung von Fake News hat gezeigt,
dass die momentane rechtliche und ethische Bewertung noch erheblichen Anpassungsbedarf 
aufweist. Die strafrechtliche Verantwortung liegt derzeit fast immer beim Verbreiter 
der Desinformation. Die rechtliche Verantwortung des Entwicklers ist noch kaum 
geregelt. Bestehende Gesetze wie das Strafgesetzbuch oder das Telemediengesetz 
decken zwar klassische Fälle von Desinformation ab, sind aber auf \ac{KI}-generierte 
Inhalte kaum anwendbar.
\end{document}